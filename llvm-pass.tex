\documentclass{report}
\usepackage{minted}
\usepackage[a4paper,margin=1in,footskip=0.25in]{geometry}
\usepackage[hidelinks]{hyperref}

\begin{document}
\chapter *{\centering LLVM}

When source code is written by a programmer in a human readable form, it needs to be translated to machine language. There are three main concepts in achieving that goal:
\begin{itemize}
	\item Don't compile. Instead, interpret code every time it's executed.
	\item Compile to byte code common for various devices and then translate to machine code on a target.
	\item Compile to machine code.
\end{itemize}
In our focus is the third approach.
\\ \\
The \textbf{LLVM Project} is a collection of modular and reusable compiler and toolchain technologies like clang, lldb, llc, lld, etc.
In order to compile source code into binary form, compilers perform various analyzes:
\begin{itemize}
	\item Lexical - Splits source code into tokens (parentheses, identifiers, constants, key words etc.)
	\item Syntax - Checks if braces are correctly paired, is there comma, semicolon or similar sign missing...
	\item Semantic - Code is syntactically valid, but doesn't make sense. For instance, there is no point in writing code in function after it returns.    
\end{itemize}
Yet, many errors stay undetected by both compiler and a programmer.
In order to reduce them, LLVM compiler uses certain optimizations called \textbf{passes}.

\section *{Build Configuration}
For detailed building of LLVM, one may consult official documentation. 
One of possibilities:
\begin{minted}[breaklines]{bash}
git clone 'https://github.com/llvm/llvm-project.git'
cd llvm-project
mkdir build && cd build
cmake -G Ninja -DLLVM_ENABLE_PROJECTS="clang;lld;lldb;llvm" -DCMAKE_INSTALL_PREFIX="$HOME/builds/llvm-release" -DLLVM_CCACHE_BUILD='ON' -DCMAKE_BUILD_TYPE="Release" -DENABLE_ASSERTIONS="ON" ../llvm
ninja
ninja install
\end{minted}

\section *{Implementing a Custom Pass}
Let's suppose we have the following code saved in file \textit{f-never-returns.c}. Note that function f never returns because the \textit{for} loop never ends: unsigned variable will never be less than zero.
\begin{minted}{c}
#include <iostream>
	
using namespace std;
	
unsigned f()
{
	unsigned cnt = 0, i = 1000;
	for(; i >= 0; i--)
	    cnt++;
	
	return cnt;
}
	
int g()
{
	int a = 5;
	for(int j = 0; j < 10; j++)
	    a *= -1;
	
	return a;
}
	
int main()
{
	int rv_g = g();
	cout << "Main: g returned " << rv_g << endl;
	unsigned rv_f = f();
	cout << "Main: f returned" << rv_f << endl;
	return 0;
}
\end{minted}
It would be nice if we'd have compiler optimization that detects such errors. \\The following steps implement solution. It is supposed that you have already cloned llvm-project repo and built llvm.
\\ \\
Add the following content to the file
\textit{llvm-project/llvm/include/llvm/Transforms/Utils/FunCantReturn.h}:
\begin{minted}{c}
#ifndef LLVM_TRANSFORMS_FUNCANTRETURN_H
#define LLVM_TRANSFORMS_FUNCANTRETURN_H

#include "llvm/IR/PassManager.h"

namespace llvm {

class FunCantReturnPass : public PassInfoMixin<FunCantReturnPass> {
	public:
	PreservedAnalyses run(Function &F, FunctionAnalysisManager &AM);
	};
} // namespace llvm

#endif // LLVM_TRANSFORMS_FUNCANTRETURN_H
\end{minted}
and the following to the \textit{llvm-project/llvm/lib/Transforms/Utils/FunCantReturn.cpp}:
\begin{minted}{c}
#include "llvm/Transforms/Utils/FunCantReturn.h"

using namespace llvm;

PreservedAnalyses FunCantReturnPass::run(Function &F,
FunctionAnalysisManager &AM) {
	if (F.doesNotReturn())
		errs() << "WARNING: " << F.getName() << " can't return." << "\n";
	return PreservedAnalyses::all();
}
\end{minted}
Register your pass by adding the following to \textit{llvm-project/llvm/lib/Passes/PassRegistry.def} in the \textit{FUNCTION\_PASS} section:
\begin{minted}{c}
FUNCTION_PASS("fun-cant-return", FunCantReturnPass())
\end{minted}
Add the proper include directive in llvm-project/llvm/lib/Passes/PassBuilder.cpp:
\begin{minted}{c}
#include "llvm/Transforms/Utils/FunCantReturn.h"
\end{minted}
and finally, add new pass to \textit{llvm-project/llvm/lib/Transforms/Utils/CMakeLists.txt}:
\begin{minted}{c}
FunCantReturn.cpp
\end{minted}
Build LLVM optimizer called \textit{opt} which performs passes on \textbf{Intermediate Representation} (IR) code:
\begin{minted}{bash}
cd llvm-project/build
ninja opt
ninja install # just if no errors reported
\end{minted}
After installation is complete, we can see our pass is available in just built \textit{opt} version:
\begin{minted}{bash}
build/opt -print-passes | grep fun-cant-return

# following line will be printed only if our pass has 
# successufully added to opt. If not, return back and try to
# find error.
fun-cant-return
\end{minted}
Let's test it on our code:
\begin{minted}{sh}
# get IR (.ll) from source
build/clang -emit-llvm -S f-cant-return.c -g -O2

# perform pass on IR code
build/opt f-cant-return.ll -passes=fun-cant-return 

# opt's output. '_Z1fv' is name of function f in produced .ll code
# main never returns because it calls f which has an infinite for-loop
WARNING: _Z1fv can't return.
WARNING: main can't return.
\end{minted}
One may write another pass in a similar way. The most important part is \textit{run} method in \textit{llvm-project/llvm/lib/Transforms/Utils/*.cpp} file because it defines what pass actually does. Registering, rebuilding opt and other steps remain the same.









































\end{document}

